% !TeX spellcheck = es_ES
\documentclass[a4paper, 11pt]{article}
\usepackage[table,xcdraw]{xcolor}
\usepackage{graphicx}
\usepackage{hyperref}
\usepackage{mathtools}
\usepackage{amsmath}
\usepackage[spanish, activeacute]{babel} %Definir idioma español
\usepackage[utf8]{inputenc} %Codificacion utf-8

\usepackage{geometry}
\geometry{left=2.5cm,right=2.5cm,top=2.5cm,bottom=2.5cm}

\DeclareMathOperator*{\argmax}{arg\,max}

\title{\Large{\textbf{Detección de caras basada en Redes Neuronales}}}

\author{\textit{Francisco Rubin Capalbo}\\
		Universidad Politécnica de Valencia }

\date{\today}

\begin{document}
    
    \maketitle
    \section{Introducción}
    
    \section{}


- dataset caras / nocaras
	- extraccion (google images, celebA)
	 	- mencionar posible problema al usar haar para extraer las caras de celebA
- clasificador
	- data augmentation
	- data normalization
		- las imágenes pasadas al clasificador son en blanco y negro  y estan normalizadas
		- también resizeadas a unas dimensiones fijas (24x24)
	- diferentes modelos / arquitecturas
		- bce loss
		- descripciones modelos
		- resultados de cada uno de ellos (tabla)
			- mencionar que todas las pruebas fueron realizadas en un dataset pequeño, incluyendo sólo un 5\% de los datos
- detector
	- downscale image to a max witdth and height to reduce number of crops
	- extracción de crops 
		- mencionar que al pasar los crops a memoria antes de clasificar mejoró de 180 segundos a 10 segundos
	- clasificación de crops en cara/no cara
	- pintar rectángulos en imagen final
	- tutorial de como usar el detector (para q lo pueda probar el profe)
- Trabajo futuro
	- user selective search para selección de crops
	- probar con depthwise separable convolutionals para mejorar velocidad
	- aumentar los saltos de zoom y translation para las seleccion de crops usando el modelo con data augmentation potenciada
	

\begin{thebibliography}{50}
	
	\bibitem{demo} 
	\href{url }{name}
	\bibitem{demo} 
	\href{url }{name}
	\bibitem{demo} 
	\href{url }{name}
	\bibitem{demo} 
	\href{url }{name}
	\bibitem{demo} 
	\href{url }{name}
	\bibitem{demo} 
	\href{url }{name}
	
\end{thebibliography}

\end{document}